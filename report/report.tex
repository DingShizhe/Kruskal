\documentclass[UTF8]{ctexart}
% include graphics
\usepackage{graphics}

% list paragraph
\usepackage{listings} 

% layout setting
\usepackage{geometry}

% generate frame
\usepackage{mdframed}

% random get paragraph
\usepackage{lipsum}

% layout
\geometry{a4paper,scale=0.75}

% title
\title{\bf 数据结构实验报告}
\author{\it 丁诗哲\ \tt{dingshizhe@gmail.com}}
\date{\today}

% new command
% \newcommand{\ok}{dingshizhe}

\begin{document}
\maketitle

\textbf{题目}:编制一个利用Kruskal算法生成一个网的最小生成树的程序。
\
\newpage

\begin{section}{需求分析}

\begin{subsection}{基本数据类型}
	用一个含有两个静态数组变量和两个int变量的结构表示一个图。其中的一个数组存储图的边,另一个数组存储图的节点;其中的一个int变量存储图中边的个数,另一个存储图中节点的个数。\\
	\par{一个节点的变量类型为int,一条边的变量类型为一个结构——这个结构中存有这条边带有的两个节点和边的权。}
\end{subsection}

\begin{subsection}{标准输入}
	用户需要标准化输入一个图,需要按照提示依次输入下列数据:点数,边数,边(包括两个点和一个权值,分别用空格隔开)。每次输入用换行分开。
\end{subsection}

\begin{subsection}{标准输出}
	数据输入完毕后,用户将得到一个最小生成树的标准化输出,输出包含下列数据:点数,边数,点的列表以及边的列表。
\end{subsection}

\begin{subsection}{其他}
	本程序需要另外一个数据结构——并查集(MFset)。
	\par{我们用一个结构描述并查集。此结构包含一个int型的静态数组和int型的描述数组大小的量。数组的第i个值描述i在哪个集合中。它的值初始为i。}
	\par{我们用并查集描述Kruskal算法运行过程中图的节点所属的连通分量。}


\end{subsection}


\end{section}


\begin{section}{概要设计}

\begin{subsection}{抽象数据结构定义}

\textbf{\\图(Graph)的抽象数据结构定义:}

\begin{mdframed}[everyline=true]
\begin{lstlisting}
  ADT Graph{
    数据对象 V: V是具有相同特性的数据元素的集合,称为顶点集;
    数据关系 R: R = {VR},VR = {<v,w>|v,w属于V},R是V×V的子集,称为边集;
    基本操作 P: 
      Create_graph(&G, V, VR)
        初始条件:图G存在,V是顶点集,VR是边集。
        操作结果:以V和VR为条件生成一个图。
      Destroy_graph(&G)
        初始条件:图G存在。
        操作结果:销毁图G。
      Get_vex(&G, v)
        初始条件:图G存在,边v存在。
        操作结果:得到G中v的权重。
      Put_vex(&G, v, weight)
        初始条件:图G存在,边v存在。
        操作结果:给边v赋权weight。
      Add_vex(&G, v)
        初始条件:图G存在,边v不存在。
        操作结果:给G加边v。
      Initial_mini_gtree(&G)
        初始条件:图G存在。
        操作结果:创建一个图,大小和G相同但是不含边。
              (为最小生成树做准备)
      Print_graph(&G)
        初始条件:图G存在。
        操作结果:在终端打印出图G的信息。
  }
\end{lstlisting} 
\end{mdframed}

\textbf{\\并查集(MFset)的抽象定义:}

\begin{mdframed}[everyline=true]
\begin{lstlisting}
  ADT MFset{
    数据对象 S: S是一个MFset,则它的元素是一系列子集S[i](i=1,2...m),
        每个子集的元素是{0,1...n}中的一个整数。
    数据关系 R: S[1]、S[2]...S[m]的并集为初始定义的全集。
    基本操作 P:
    	 Initial(&S, n, x1, x2...xn)
    	   操作结果:
    	 Find(&S, x)
    	   初始条件:S存在,x是S子集的一个元素。
    	   操作结果:返回x所在的S[i]。
    	 Merge(&S, i, j)
    	   初始条件:S存在。
    	   操作结果:将i和j所在的S的子集合并。
    	 Destroy(&S)
    	   初始条件:S存在。
    	   操作结果:将S销毁。
    	 IsSame(&S, i, j)
    	   初始条件:S存在,i和j在S的子集中。
    	   操作结果:返回i和j是否在同一个子集中。
    	 PrintMfset(&S)
    	   初始条件:S存在。
    	   操作结果:打印出S的值。
}
\end{lstlisting}
\end{mdframed}

\end{subsection}


\begin{subsection}{程序构成}

本程序包含四个模块:主程序模块、图结构模块、并查集结构模块、克鲁斯卡尔算法实现模块。

\textbf{\\主程序模块:}在main.c中,程序结构如下:

\begin{mdframed}[everyline=true]
\begin{lstlisting}[language=c]
	int main(){
	   初始化图G;
	   打印图G;
	   对G进行Kruskal算法;
	   打印图;
	   销毁图;
	   返回0;
	}
\end{lstlisting}
\end{mdframed}


\textbf{\\图结构模块:\\}\indent{在文件graph.h和graph.c中,graph.h声明了图的结构和相关操作。graph.c中给出了操作的具体定义。\\}
\indent{graph.h中声明的操作有:}

\begin{mdframed}[everyline=true]
\begin{lstlisting}[language=c]
    Graph create_graph(void);
    int delete_graph(Graph G);
    int add_edge(edge tmp, Graph G);
    Graph create_test_graph(void);
    Graph initial_mini_gTree(Graph G);
    int print_graph(Graph G);
\end{lstlisting}
\end{mdframed}

\indent{其中$Graph\ create\_graph(void)$是由用户输入数据生成一个图,$Graph create\_test\_graph(void)$是读取一个文件的数据生成一个特定的测试图,该文件$(test.txt)$已经被写好。其他操作分别对上面Graph 抽象ADT中的各种操作。}



\textbf{\\并查集模块:\\}\indent{在文件mfset.h和mfset.c中,mfset.h声明了图的结构和相关操作。mfset.c中给出了操作的具体定义。}
\indent{mfset.h中声明的操作有:}

\begin{mdframed}[everyline=true]
\begin{lstlisting}[language=c]
    MFset create_mfset(int size);
    int find(int x, MFset MF);
    int is_insame(int x, int y, MFset MF);
    int merge(int x, int y, MFset MF);
    int delete_mfset(MFset MF);
    int print_mfset(MFset MF);
\end{lstlisting}
\end{mdframed}


\indent{它们分别对上面MFset 抽象ADT中的各种操作。}



\textbf{\\克鲁斯卡尔算法实现模块:}在文件kruskal.c中。

\par{文件的调用关系如下:}

\includegraphics[height=4cm]{"fig1.png"}

main.c调用graph.h和mfset.h中声明的函数,也调用kruskal.c中的函数。kruskal.c需要graph.h和mfset.h中声明的函数。

\end{subsection}

\end{section}

\begin{section}{详细设计}


\begin{subsection}{图的结构类型}

% \textbf{create_graph}

\begin{mdframed}[everyline=true]
\begin{lstlisting}[language=c]
  // 边结构
  struct edge{
    int from, to;
    int weight;
  };

  typedef struct edge edge;

  // 图结构(结点集合,边集合,结点数,边数)
  struct Graph_box{
    char vertices[30];
    edge edges[100] ;
    int vexnum, edgenum;
  };

  typedef struct Graph_box * Graph;

\end{lstlisting}
\end{mdframed}

\end{subsection}

\begin{subsection}{对图的一些操作}

\par{下面是创建一个图$(Graph\ create\_graph(void))$的程序代码,包含提示输入数据类型,获取标准输入,根据标准输入创建无边图,依次加入边。}

\begin{mdframed}[everyline=true]
\begin{lstlisting}[language=c]
  // 按输入建立一个图
  Graph create_graph(void){

    Graph G = (Graph) malloc(sizeof(struct Graph_box));
    if(!G) return NULL;

    printf("输入顶点数\n");
    scanf("%d", &(G->vexnum));
    printf("输入边数\n");
    scanf("%d", &(G->edgenum));

    int i;
    for(i=0; i<G->vexnum; i++)
      G->vertices[i] = 'A' + i;
      G->vertices[i] = '\0';
      printf("输入边\n");

      // 初始化
      for(i=0; i<100; i++)
      G->edges[i].weight = 0;

      // 临时
      edge tmp;

      // 新的边插入原有的边列表,保持按权重降序
      for(i=0; i<G->edgenum; i++){
        scanf("%d %d %d", &(tmp.from), &(tmp.to), &(tmp.weight));
        add_edge(tmp, G);
    }
    return G;
  }
\end{lstlisting}
\end{mdframed}

\par{下面是在图$G$中加入一条边$tmp$的程序$(int add_edge(edge\ tmp, Graph\ G))$代码。需要注意的是,加入边的过程中,每次加不仅需要判断和已有的边是否重复,如果不重复、还需要加入后边的数组按照边的权重保持降序。这是为了运行Kruskal算法的方便。\\}

\begin{mdframed}[everyline=true]
\begin{lstlisting}[language=c]
    int add_edge(edge tmp, Graph G){
      if(!G) return -1;

      // 加入的边不能和已有的边重复
      for(int i=0; i<G->edgenum; i++)
        if((tmp.from==G->edges[i].from && tmp.to==G->edges[i].to)||
        (tmp.from==G->edges[i].to&&tmp.to==G->edges[i].from))
        return -1;

      // 保证加入边后列表按照权重降序
      for(int j=0; j<=G->edgenum; j++)
        if(tmp.weight > G->edges[j].weight){
          for(int k=G->edgenum; k>j; k--)
            G->edges[k] = G->edges[k-1];
            G->edges[j] = tmp;
            break;
          }
      G->edgenum++;
      return 0;
    }
\end{lstlisting}
\end{mdframed}

\end{subsection}

\begin{subsection}{并查集的结构类型}

\begin{mdframed}[everyline=true]
\begin{lstlisting}[language=c]
    // 并查集
    struct mfset{
      int set[MAX_NUM];
      int size;
    };

    typedef struct mfset * MFset;
\end{lstlisting}
\end{mdframed}

\end{subsection}



\begin{subsection}{对并查集的部分操作}
\par{下面是将并查集$MF$中的$i$和$j$所在的子集合并的程序$(int merge(int\ x,\ int\ y,\ MFset\ MF))$。现在用$f(i)$表示MF结构内的数组第$i$位的值。}
\par{我们在前面的定义中规定,$f(i)$表示其所在的集合,不同的$i$和$j$对应的$f(i)$、$f(j)$相等,意味着他们在同一个集合中。我们将$i$和$j$所在的集合合并,只需将和$f(i)$以及$f(j)$相等的所有的位置$x$的$f(x)$都设为相同的值即可。但是我需要设置他们的值和其他集合的值不相等,考虑到数组初始化的时候$f(i)=i$,再$merge$操作中我们这样约定:将并查集$MF$中的$i$和$j$所在的子集合并时,把所有和$i$值相同的元素$x$和$j$值相同的元素$y$的$f(x)$、$f(y)$全部设为$\min(f(i), f(j))$。\\}

\indent{实现代码如下:}

\begin{mdframed}[everyline=true]
\begin{lstlisting}[language=c]
    int merge(int x, int y, MFset MF){
      if(MF->set[x] == MF->set[y]) return 0;
      int tmp;
      if(MF->set[x] < MF->set[y]){
        tmp = MF->set[y];
        for(int i=0; i<MAX_NUM; i++)
          if(MF->set[i] == tmp)
            MF->set[i] = MF->set[x];
      }
      else{
        tmp = MF->set[x];
        for(int i=0; i<MAX_NUM; i++)
        if(MF->set[i] == tmp)
          MF->set[i] = MF->set[y];
      }
      return 0;
    }
\end{lstlisting}
\end{mdframed}

\end{subsection}


\begin{subsection}{Kruskal算法的实现}
\par{Kruskal算法是一种用来寻找最小生成树的算法。在剩下的所有未选取的边中,找最小边,如果和已选取的边构成回路,则放弃,选取次小边。下面是算法在本程序中的实现:\\}

\begin{mdframed}[everyline=true]
\begin{lstlisting}[language=c]
    Graph kruskal(Graph G){
      if(!G) return NULL;
      // 初始化最小生成树
      Graph T = initial_mini_gTree(G);

      int from, to;
      // 创建并查集
      MFset MF = create_mfset(G->vexnum);

      // 由于边按照权降序,我们从最后一条边开始循环
      for(int i=G->edgenum-1; i>=0; i--){
        from = G->edges[i].from;
        to = G->edges[i].to;

        // 如果在同一个子集中,进行下一轮循环
        if(is_insame(from, to, MF))
          continue;
        // 否则也就是说他们不在同一个子集中
        // 那么将他们所在的子集合并
        else{
          add_edge(G->edges[i], T);
          merge(from, to, MF);
        }
      }
      delete_mfset(MF);
      return T;
    }
\end{lstlisting}
\end{mdframed}

\par{此算法根据输入的图G,生成一个节点数相等但无边的图T,这是初始化了的最小生成树。然后以输入的图G的节点数为大小创建一个并查集MF。}
\par{接着从G的最小边(最后一条边)开始进行循环:如果边的两个节点在并查集中属于同一个子集,那么继续下一轮循环;如果不在,那么把这条边加入图T,并把并查集中这两个节点所在的子集合并。}
\par{所有的边遍历完,程序结束,返回图T。\\}

\end{subsection}


\begin{subsection}{主程序的测试代码}
\par{下面是主程序的测试代码:\\}

\begin{mdframed}[everyline=true]
\begin{lstlisting}[language=c]
    int main(int argc, char const *argv[])
    {
      // 如果有命令行参数,那就生成特定的一个测试用例
      if(argc==2)
        Graph G = create_test_graph();
      // 否则由用户来输入生成图
      else
        Graph G = create_graph();
      print_graph(G);
      Graph T = kruskal(G);
      printf("最小生成树是:\n");
      print_graph(T);
      delete_graph(G);
      delete_graph(T);
      return 0;
    }
\end{lstlisting}
\end{mdframed}

\par{此程序首先根据用户输入或者读取文件创建一个图G并打印,然后调用$Graph\ kruskal(Graph\ G)$函数,该函数生成最小生成树T并打印。}

\end{subsection}

\begin{subsection}{具体的函数调用关系}

\par{具体函数的调用关系如下图所示:}

\includegraphics[height=9cm]{"fig3.png"}

\end{subsection}

\end{section}




\end{document}


% [language=c, frame=shadow]

% \begin{subsection}{ok}
% \begin{mdframed}[everyline=true]
% \lipsum
% \end{mdframed}
% \end{subsection}